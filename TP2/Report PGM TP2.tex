\documentclass[11pt,a4paper]{article}

\usepackage{etex}
\usepackage[applemac]{inputenc}
\usepackage{latexsym}
\usepackage{graphicx}
\usepackage[francais]{babel}
\usepackage{amsmath,amssymb}
\usepackage{pstricks,pst-plot}
\usepackage{calc}
\usepackage{multicol}
\usepackage{fancyhdr}
\usepackage{lastpage}
\usepackage[T1]{fontenc}
\usepackage{stmaryrd}
\usepackage{float}
\pagestyle{plain}
\usepackage[top=2cm, bottom=2cm, left=3cm, right=3cm]{geometry}

\usepackage{tikz}
\usetikzlibrary{arrows}

\newcommand{\indep}{\rotatebox[origin=c]{90}{$\models$}}

	\title{Probabilistic Graphical Models : Assignment 2}
	\author{Mathurin \textsc{Massias} \and Cl�ment \textsc{Nicolle}}
	\date{\today}

\begin{document}
	
	\maketitle

\section{Distributions factorizing in a graph}

\textit{(a)}
\\i. This is a covered edge, for example :
\\
\begin{center}
\begin{tikzpicture}[->,>=stealth',shorten >=1pt,auto,node distance=2cm,
thick,main node/.style={circle,draw,font=\sffamily\Large\bfseries}]

\node[main node] (1) {};
\node[main node] (2) [below of=1] {i};
\node[main node] (3) [right of=2] {j};
\node[main node] (4) [below of=2] {};

\path[every node/.style={font=\sffamily\small}]
	(1) edge node {} (2)
	edge node {} (3)
	(2) edge node {} (3)
	(4) edge node  {} (2)
	edge node {} (3);

\end{tikzpicture}
\end{center}

The parents of the node $j$ are $\left\lbrace  i + parents\:of\:i \right\rbrace$. We now reverse $i \rightarrow j$. Let show that $\mathcal{L}(G) = \mathcal{L}(G')$
\\
\\
If $p(x) \in \mathcal{L(G)}$ we have :
\\$p(x) = \mathop{\Pi}\limits_{k=1}^N p(x_k | x_{\pi_k}) = \mathop{\Pi}\limits_{k \not\in \left\lbrace i,j \right\rbrace} p(x_k | x_{\pi_k}) \times p(x_i | x_{\pi_i}) \times p(x_j | x_{\pi_j}) $
\\
But, with $\pi_j = \pi_i \cup \left\lbrace i \right\rbrace$, we have :
\\
$p(x_i | x_{\pi_i}) \times p(x_j | x_{\pi_j}) = p(x_i | x_{\pi_i}) \times p(x_j | x_{\pi_i}, x_i) = p(x_i | x_{\pi_i}) \times p(x_j | x_{\pi_i})$
\\ \-\hspace{1cm} because $x_i$ depends on $x_{\pi_i}$
\\And if we reverse  $i \rightarrow j$, $x_{\pi_i}$ becomes $x_{\pi_j}$ and the value of the product above will be the same.
\\
\\
So $\mathcal{L}(G) = \mathcal{L}(G')$
\\
\\
ii. Let $G$ be a directed tree without v-structures and $G'$ the undirected tree with the same edges (the symmetrized graph). Let show that $\mathcal{L}(G) = \mathcal{L}(G')$
\\
\\
As there is no v-structures in $G$, it means that the cliques of $G'$ are only composed of couples : $C= \left\lbrace (x_i, x_{\pi_i}) \right\rbrace$
\\Each $x_i$ has a single parent.
\\Thus we have $p(x) = \mathop{\Pi}\limits_{k=1}^N p(x_k | x_{\pi_k}) = \frac{1}{Z} \mathop{\Pi}\limits_{c \in C} \psi_c (x_c) $ with $\psi_c (x_c) \: \alpha \: p(x_c | x_{\pi_c})$
\\
\\
So $\mathcal{L}(G) = \mathcal{L}(G')$
\\
\\
\textit{(b)}
\\According to (a)ii., if the number of nodes is 1 or 2 we have no v-structures so $\mathcal{L}(G) = \mathcal{L}(G')$.
For 3 nodes, we need to oblige to have a v-structure in the directed graph. If we chose an undirected graph where the three nodes ae linked, we will compulsorily have a v-structure, as below :

\begin{center}
\begin{tikzpicture}[->,>=stealth',shorten >=1pt,auto,node distance=2cm,
thick,main node/.style={circle,draw,font=\sffamily\Large\bfseries}]

\node[main node] (1) {1};
\node[main node] (2) [below of=1] {2};
\node[main node] (3) [right of=2] {3};

\path[every node/.style={font=\sffamily\small}]
(1) edge node {} (2)
	edge node {} (3)
(2) edge node {} (3);

\end{tikzpicture}
\end{center}

So the symmetrized graph is composed of only one maximal clique, but we have $p(x) = p(x_1) \times p(x_2 | x_1) \ times p(x_3 | x_1, x_2)$, which cannot be written as $\psi(x_1, x_2, x_3)$

\section{d-separation}

\textit{(a)}
\\We saw this definition in class : A and B are d-separated by S if all chains from $a \in A$ to $b \in B$ (in the symmetrized graph) are blocked at least at one node in S. 
\\ Here we know that S separates A and B in the moral graph $G_M$, which includes the symmetrized graph. So S also separates A and B in the symmetrized graph. Therefore, A and B are d-separated by S in G.
\\
\\\textit{(b)}
\\Let's go back to the definition.
\\A and B are d-separated by S means that all chains from $a \in A$ to $b \in B$ are blocked at one node in S.
\\A and S are d-separated by T means that all chains from $a \in A$ to $s \in S$ are blocked at one node in T.
\\So all chains from $a \in A$ to $b \in B$ will go through S, and then through T before that.
\\We have that all chains from $a \in A$ to $b \in B$ are blocked at one node in T. In other terms, A and B are d-separated by T.
\\
\\\textit{(c)}
\\- $X_{\left\lbrace 1,2 \right\rbrace} \indep X_4 | X_3$ : False
%not sure and can't really tell why
\\
\\- $X_{\left\lbrace 1,2 \right\rbrace} \indep X_4 | X_5$ : False
%idem
\\
\\$X_1 \indep X_6 | X_{\left\lbrace 2,4,7 \right\rbrace}$ : True
\\Indeed, as $X_{\pi_6} = X_7$ we have $X_6 | X_{\left\lbrace 2,4,7 \right\rbrace} = X_6 \indep X_7$, and we are in the scheme of a Markov chain.

\end{document}